\documentclass[	DIV=calc,%
							paper=a4,%
							fontsize=12pt,%
							twocolumn]{scrartcl}	 					% KOMA-article class

\usepackage{lipsum}													% Package to create dummy text
\usepackage[english]{babel}										% English language/hyphenation
\usepackage[protrusion=true,expansion=true]{microtype}				% Better typography
\usepackage{amsmath,amsfonts,amsthm}					% Math packages
\usepackage[pdftex]{graphicx}									% Enable pdflatex
\usepackage[svgnames]{xcolor}									% Enabling colors by their 'svgnames'
\usepackage[hang, small,labelfont=bf,up,textfont=it,up]{caption}	% Custom captions under/above floats
\usepackage{epstopdf}												% Converts .eps to .pdf
\usepackage{subfig}													% Subfigures
\usepackage{booktabs}												% Nicer tables
\usepackage{fix-cm}													% Custom fontsizes
\usepackage{natbib}                                                 % Bibliography
\setcitestyle{authoryear,open={(},close={)}}            %Citation-related commands
\usepackage{hyperref}
\hypersetup{
    colorlinks=true,
    linkcolor=blue,
    filecolor=blue,      
    urlcolor=blue,
    citecolor = blue,
    pdftitle={RTU DESS Format},
    pdfpagemode=FullScreen,
    }
\usepackage{float}

%%% Custom sectioning (sectsty package)
\usepackage{sectsty}													% Custom sectioning (see below)
\allsectionsfont{%															% Change font of al section commands
	\usefont{OT1}{phv}{b}{n}%										% bch-b-n: CharterBT-Bold font
	}

\sectionfont{%																% Change font of \section command
	\usefont{OT1}{phv}{b}{n}%										% bch-b-n: CharterBT-Bold font
	}



%%% Headers and footers
\usepackage{fancyhdr}												% Needed to define custom headers/footers
	\pagestyle{fancy}														% Enabling the custom headers/footers
\usepackage{lastpage}	

% Header (empty)
\lhead{}
\chead{}
\rhead{}
% Footer (you may change this to your own needs)
\lfoot{\small \usefont{OT1}{lmr}{c}{n} \textcolor{blue}{Short Research Title}}
\cfoot{}
\rfoot{\footnotesize page \thepage\ of \pageref{LastPage}}	% "Page 1 of 2"
\renewcommand{\headrulewidth}{0.0pt}
\renewcommand{\footrulewidth}{0.4pt}

\usepackage[font=small,format=plain,labelfont=bf,
textfont=normal,singlelinecheck=false]{caption}

%%% Creating an initial of the very first character of the content
\usepackage{lettrine}
\newcommand{\initial}[1]{%
     \lettrine[lines=3,lhang=0.3,nindent=0em]{
     				\color{DarkBlue}
     				{\textsf{#1}}}{}}



%%% Title, author and date metadata
\usepackage{titling}															% For custom titles

\newcommand{\HorRule}{\color{Black}%			% Creating a horizontal rule
									  	\rule{\linewidth}{2pt}%
										}
%%begin novalidate
\pretitle{\vspace{-60pt} \begin{flushleft} \HorRule 
				\fontsize{50}{50} \usefont{OT1}{phv}{b}{n} \color{DarkBlue} \selectfont 
				}
\title{Research Title}					% Title of your article goes here
\posttitle{\par\end{flushleft}\vskip 0.5em}

\preauthor{\begin{flushleft}
					\fontsize{20}{50} \lineskip 0.5em \usefont{OT1}{lmr}{b}{ol} \color{DarkRed}}
\author{Lee Dong Min$^{1, 2,\star}$, Kim Moon Bin$^{1,3}$, Yoon Sanha$^{1,4}$, \newline Kim Myung Jun$^{1,5}$, : Park Jin Woo$^{1,6}$, and Park Min Hyuk$^{1,7}$ }											% Author name goes here
\postauthor{\fontsize{15}{10} \usefont{OT1}{lmss}{bx}{ol} \newline \newline \color{Black} 
								$^{1}$Mercury
        \newline                $^{2}$Venus
        \newline                $^{3}$Earth
        \newline                $^{4}$Mars
        \newline                $^{5}$Jupiter
        \newline                $^{6}$Saturn
        \newline                $^{7}$Neptune
					\par\end{flushleft}\HorRule
     
     \fontsize{10}{50} \usefont{OT1}{pmt}{b}{n} \color{black}
    $^\star$Corresponding Author: leedongmin@astro.aroha.com}
%%end novalidate
\date{}																% No date



%%% Begin document
\begin{document}
\maketitle
\thispagestyle{fancy} 			% Enabling the custom headers/footers for the first page 
% The first character should be within \initial{}
\section*{Abstract}
\initial{\textbf{T}}\textbf{
	he goal of the first courselab of the EE2C11 Integrated Circuits course was to simulate the behaviour of a MOS transistor in a 180nm node and then extract the unified model parameters from this simulation, plot them and compare with the simulation data. The main objective of this assigment was to learn and understand the Unified Model and methoeds behind trasistor modeling within this model. }

\section*{Introduction}
\; \; %% '\; \;' is required for indention
According to the manual of this courselab <refrence> the main educational goals of this assigment were to:
\begin{itemize}
	\item understand the behaviour of MOS integrated circuits and simple methodes of modeling of this behaviour;
	\item practice the process of performing, describing, and analyzing the experiment and its results;
	\item enchance our report writing, and data gathering skills.
\end{itemize}

While the last objective was achieved mostly through the process of report writing, the first two were achieved through the actuall experiment. Its entire process consisted of a few steps.

\begin{itemize}
	\item Firstly the simulation in LTSpice was performed, chapter 2 <link> expands on this further


\section*{Methodology}
\; \; %% '\; \;' is required for indention
\lipsum[6]
\begin{itemize}
	\item \lipsum[7] 
	\item \lipsum[7] 
	\item \lipsum[7] 
\end{itemize}

\lipsum[6]

\section*{Results}
\; \; %% '\; \;' is required for indention
\lipsum[7] (see Table~\ref{Tab1}) 

\subsection*{Result 1}
\; \; %% '\; \;' is required for indention 
\lipsum[7] (see Figure~\ref{Figure 1})

\begin{table}
\caption{Caption of table}
\centering
	\begin{tabular}{llr}
		\toprule
		\multicolumn{2}{c}{Name} \\
		\cmidrule(r){1-2}
			First name & Last Name & Grade \\
		\midrule
			John & Doe & $7.5$ \\
			Richard & Miles & $2$ \\
		\bottomrule
	\end{tabular}
 \label{Tab1}
\end{table}


\begin{figure}[H]
    \centering
    \includegraphics[width=\columnwidth]{7.png}
    \caption{Sphere Mesh}
    \label{Figure 1}
\end{figure}

\section*{Discussion}
\; \; %% '\; \;' is required for indention
\lipsum[7]

\begin{description}
	\item[First] \lipsum[30] 
	\item[Last] \lipsum[30] 
\end{description}

\lipsum[3] 
According to \citet{Ref1} and \citet{Ref2}, %For multiple citations use ',' in between references
\lipsum[2] 

\begin{figure}[H] %'[H]' forces the figure to stay in the relative position of the text it is located
    \centering
    \includegraphics[width=\columnwidth]{13.png}
    \caption{Blocks}
    \label{Figure 2}
\end{figure}

\lipsum[5](see Figure~\ref{Figure 2,Figure 3})

\section*{Conclusion}
\; \; %% '\; \;' is required for indention
\lipsum[8]

\lipsum[8]

\lipsum[8]

\section*{Acknowledgments}
\; \; %% '\; \;' is required for indention
\lipsum[1]

\section*{Author Contributions}
\; \; %% '\; \;' is required for indention
The author did not receive any external help to write this paper/ The authors provided the following contributions: Author 1, Author 2 - Statistical Analysis, Author 3 - Data Visualization, etc.

%% Alphabetical References
%\bibliographystyle{unsrtnat}   %No abbreviation of First Name
\bibliographystyle{abbrvnat}    %Abbreviation of First Name
\bibliography{references} %% To edit the references copy paste the BibTex citation format in the references.bib file in the left

\section*{Appendices}
\lipsum[1]


\begin{figure}[H] %'[H]' forces the figure to stay in the relative position of the text it is located
    \centering
    \includegraphics[width=\columnwidth]{catto.png}
    \caption{Scary Shark Catto}
    \label{Figure 3}
\end{figure}

    
\lipsum[1]

\subsection*{Appendix 1}

\lipsum[1]

\lipsum[1]

\lipsum[2]

\lipsum[2]

\lipsum[2]
\end{document}
